\section{Graphen}

% -----------------------------------------------------------------
%\subsection{Topologische Sortierung}

%\lstinputlisting{graphentheorie/topo-sort.cpp}

% -----------------------------------------------------------------
%\subsection{Starke Zusammenhangskomponenten (Tarjan)}

%\lstinputlisting{graphentheorie/scc.cpp}

% -----------------------------------------------------------------
%\subsection{Eulerscher Kreis}

%Bei ungerichteten Graphen existiert ein Eulerkreis genau dann, wenn alle Knoten geraden Grad haben.
%Bei gerichteten Graphen existiert ein Eulerkreis genau dann, wenn bei allen Knoten Eingangsgrad und Ausgangsgrad gleich sind.
%Nat"urlich sollte der Graph zusammenh"angend sein!

%\lstinputlisting{graphentheorie/eulerkreis.cpp}

% -----------------------------------------------------------------
%\subsection{K"urzeste Pfade (Dijkstra, Floyd-Warshall, Bellman-Ford)}

%\subsubsection{Dijkstra}

%\lstinputlisting{graphentheorie/dijkstra.cpp}

%\subsubsection{Floyd-Warshall}

%\lstinputlisting{graphentheorie/floyd-warshall.cpp}

%\subsubsection{Bellman-Ford}

%\lstinputlisting{graphentheorie/bellman-ford.cpp}

% -----------------------------------------------------------------
%\subsection{Minimaler Spannbaum (Kruskal, Prim)}

%\subsubsection{Kruskal}

%\lstinputlisting{graphentheorie/kruskal.cpp}

%\subsubsection{Prim}

%\lstinputlisting{graphentheorie/prim.cpp}

% -----------------------------------------------------------------
\subsection{Shortest Path - Dijkstra with heap}
\lstinputlisting{graphentheorie/dijkstra2.cpp}

\subsection{Maximum Bipartite Matching}
\lstinputlisting{graphentheorie/BipartiteMatching2.cpp}

\subsection{Maximaler Fluss}
\lstinputlisting{graphentheorie/Maxflow_Dinic.cpp}
% -----------------------------------------------------------------


\subsection{Minimaler Schnitt (Stoer-Wagner)}
\lstinputlisting{graphentheorie/stoer-wagner.cpp}

\subsection{Min-Cost-Max-Flow}

\lstinputlisting{graphentheorie/MincostFlow_SPFA.cpp}

\subsection{Artikulationspunkte, Br"ucken und Bikonnektivit"at (Tarjan)}
\lstinputlisting{graphentheorie/Cutnode&Bridge.cpp}


\subsection{Lowest Common Ancestor}

\lstinputlisting{graphentheorie/lca_Tarjan.cpp}

\subsection{2-SAT}

\lstinputlisting{graphentheorie/2sat.cpp}
