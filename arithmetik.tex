\section{Arithmetik und Algebra}

%\subsection{Bigints}

%\subsubsection{Bigints in C++}

%\lstinputlisting{arithmetik/bigint.cpp}

%\subsubsection{Bigints in Java}

%\lstinputlisting{arithmetik/bigint.java}

%\subsection{Schnelles Potenzieren}

%\lstinputlisting{arithmetik/potenz.cpp}

%\subsection{Endliche Summation}

%Def. Pochhammer-Symbol: \(n^{\underline{k}}=n\cdot(n-1)\cdot\ldots\cdot(n-k+1)\).
%Bei Summation "uber Polynome, dr"ucke die normalen Potenzen in dieser Schreibweise
%aus, z.B. \(3n^2-n=3n^{\underline{2}}+2n^{\underline{1}}\). Dann wie in der Analysis
%integrieren, ergibt eine Art Stammfunktion:
%\[\sum 3n^{\underline{2}}+2n^{\underline{1}} = n^{\underline{3}}+n^{\underline{2}}\]
%Wenn man das Polynom von \(1,\,\ldots,\,N\) summiert, muss man die Stammfunktion
%an der Stelle \(N+1\) und 1 auswerten:
%\[\sum_{n=1}^N 3n^{\underline{2}}+2n^{\underline{1}}
%= n^{\underline{3}}+n^{\underline{2}}\vert_{1}^{N+1}
%= (N+1)^{\underline{3}}+(N+1)^{\underline{2}} = (N+1)N^2\]
%
%Das Gegenteil vom Summationsoperator \(\sum\) ist der Differenzenoperator
%\(\Delta f(x) = f(x+1) - f(x)\). Der e-Funktion entspricht \(2^n\),
%also \(\sum 2^n = \Delta 2^n = 2^n\).

%\subsection{Matrixmultiplikation (ikj, Strassen)}

%\subsubsection{Cachefreundliche Matrixmultiplikation}

%\lstinputlisting{arithmetik/ikj.cpp}

%\subsubsection{Algorithmus von Strassen}

%\[ \begin{pmatrix} r & s \\ t & u \end{pmatrix}
%= \begin{pmatrix} a & b \\ c & d \end{pmatrix} 
%\begin{pmatrix} e & f \\ g & h \end{pmatrix} \]
%
%\begin{eqnarray}{rcl}
%p_1 & = & a(f-h) \\
%p_2 & = & (a+b)h \\
%p_3 & = & (c+d)e \\
%p_4 & = & d(g-e) \\
%p_5 & = & (a+d)(e+h) \\
%p_6 & = & (b-d)(g+h) \\
%p_7 & = & (a-c)(e+f)
%\end{eqnarray}
%
%Dann:
%
%\[ r = p_5 + p_4 - p_2 + p_6, \quad s = p_1 + p_2, \quad t = p_3 + p_4, \quad u = p_5 + p_1 - p_3 - p_7 \]

\subsection{Lineare Gleichungssysteme (LGS) und Determinanten}

%\subsubsection{Gau"s-Algorithmus}

%\lstinputlisting{arithmetik/gauss.cpp}

%\subsubsection{LR-Zerlegung, Determinanten}

\lstinputlisting{arithmetik/lr.cpp}

\subsection{Numerical Integration (Adaptive Simpson's rule)}

\lstinputlisting{arithmetik/simpson.cpp}

%\subsection{Polynome}

%\subsubsection{Newtonsche/Hermitsche Interpolation}

%Dividierte Differenzen:
%\[ [x_i,\ldots,x_j] = \frac{[x_{i+1},\ldots,x_j] - [x_i,\ldots,x_{j-1}]}{x_j - x_i} \]
%\[ [\underbrace{x_i,\ldots,x_i}_{n \mathrm{mal}}] = f^{(n-1)}(x_i)/n! \]

%Newton-Polynome:
%\[ P_i = \prod_{k=1}^{i-1} (x-x_k) \]
%Also immer ein Term weniger mulitplizieren als das neue \(x_i\)!

%Interpoliertes Polynom:
%\[ P = \sum_{i=1}^{n} [x_1,\ldots,x_i] \cdot P_i \]

%Wenn ein \(x_i\) mehrmals auftaucht, als man auch die ersten paar Ableitungen
%interpolieren will, wird dieses \(x_i\) dann halt auch im Newton-Polynom
%mehrmals dranmultipliziert.

%\subsubsection{Hornerschema}

%Seien \(f(T) = \sum_{i = 0}^{n} \alpha_i T^i\) und \(g(T) = \sum_{j = 0}^{m} \beta_j T^j\) zwei Polynome. %Setze f"ur alle \(i = n,\,\ldots,\,0\): 
%\[\gamma_i = \alpha_i - 
%\sum_{j = \max(1, m - i)}^{\min(m, n - i)} \beta_{m - j} \gamma_{i + j}\] 
%Dann ist der Rest \(r(T)\) und das Ergebnis \(s(T)\) der Division von \(f\) durch \(g\) gegeben durch: 
%\[r(T) = \sum_{i = 0}^{m - 1} \gamma_i T^i,\quad s(T) 
%= \sum_{j = m}^{n} \gamma_j T^{j-m}\]

% Polynommultiplikation via FFT
\subsection{FFT}
\lstinputlisting{arithmetik/fft.cpp}
