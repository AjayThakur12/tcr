\section{Datenstrukturen}

%\subsection{Disjunkte Mengen (Union-Find)}

%\lstinputlisting{datenstrukturen/disjoint.cpp}

%\subsection{Heap}
%In der STL gibt es \lstinline{make_heap}, \lstinline{is_heap}, \lstinline{sort_heap},
%\lstinline{push_heap} und \lstinline{pop_heap}.
%Achtung: Diese Funktionen stellen das \emph{gr"o"ste} Element an die Spitze des Heaps.
%
%\lstinputlisting{datenstrukturen/heap.cpp}

%\subsection{Fenwick-Baum}

%\lstinputlisting{datenstrukturen/fenwick.cpp}

%\subsection{Trie}

%\subsubsection{Trie mit verketteter Liste}

%\lstinputlisting{datenstrukturen/trie-list.cpp}

%\subsubsection{Trie mit Array}

%\lstinputlisting{datenstrukturen/trie-array.cpp}

%\subsection{Balancierter bin"arer Suchbaum (AVL)}

%\lstinputlisting{datenstrukturen/avl.cpp}

%\subsection{Range Minimum Query (RMQ)}

%\lstinputlisting{datenstrukturen/rmq.cpp}

%\subsection{Splay}

%\lstinputlisting{datenstrukturen/splay.cpp}

\subsection{Skew Heaps (meldable priority queue)}
\lstinputlisting{datenstrukturen/SkewHeap.cpp}

\subsection{Treap}
\lstinputlisting{datenstrukturen/treap.cpp}

\subsection{2D Range Tree}
\lstinputlisting{datenstrukturen/range_tree.cpp}

\subsection{Segment Tree}
\lstinputlisting{datenstrukturen/Segment_tree.cpp}
