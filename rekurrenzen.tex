\section{Rekurrenzen, Kombinatorik und DP}

%\subsection{Lineare Rekurrenzen (z.B. Fibonacci)}

%Lassen sich als Matrixgleichung schreiben, z.B. Fibonacci:
%\[ \begin{pmatrix} F_n \\ F_{n-1} \end{pmatrix}
%= \begin{pmatrix} 1 & 1 \\ 1 & 0 \end{pmatrix}
%\cdot \begin{pmatrix} F_{n-1} \\ F_{n-2} \end{pmatrix} \]
%F"ur gro"se \(n\) dann einfach die Matrix mittels
%schneller Potenzierung potenzieren, geht in \(\mathrm O(\log n)\).

\subsection{Urnenmodelle}

\begin{tabular}{r||c|c}
\(k\) aus \(n\) ziehen & mit Zur"ucklegen & ohne Zur"ucklegen \\
\hline
mit Beachtung der Reihenfolge & \(n^k\) & \(\binom{n}{k}\cdot k!\) \\
\hline
ohne Beachtung der Reihenfolge & \(\binom{n+k-1}{k}\) & \(\binom{n}{k}\) \\
\end{tabular}

\subsection{Binomialkoeffizienten}

\[ \binom{n}{k} = \frac{n!}{k!(n-k)!}, \quad
\binom{n}{k} = \binom{n}{n-k}, \quad
\binom{r}{k} = \frac{r}{k} \binom{r-1}{k-1}, \quad
\binom{r}{k} = \binom{r-1}{k} + \binom{r-1}{k-1} \]
\[ \binom{r}{k} = (-1)^k \binom{k-r-1}{k}, \quad
\binom{r}{m} \binom{m}{k} = \binom{r}{k} \binom{r-k}{m-k}, \quad
(x+y)^r = \sum_{k=0}^r \binom{r}{k} x^k y^{r - k} \]
\[ \sum_{0 \leq k \leq n} \binom{k}{m} = \binom{n+1}{m+1}, \quad
\sum_{k \leq n} \binom{r+k}{k} = \binom{r+n+1}{n}, \quad
\sum_k \binom{r}{k} \binom{s}{n-k} = \binom{r+s}{n} \]

\subsection{H"aufige Rekurrenzen in der Kombinatorik}

%\subsubsection{Catalansche Zahlen}
%
%Anzahl der Sehnentriangulationen eines \(n+2\)-Ecks, Anzahl der g"ultigen Klammerausdr"ucke mit \(n%\) Klammern,
%Anzahl der Pfade in einem Grid, wobei man nur eins nach rechts und eins nach unten gehen darf.
%\[ C_n = \frac{1}{n+1} \binom{2n}{n}, \quad C_{n+1} = \sum_{k=0}^{n} C_k C_{n-k} \]

\subsubsection{Stirlingsche Zahlen}

\begin{itemize}
\item 1. Art \\
Anzahl der \(n\)-Permutationen mit genau \(k\) Zyklen.
\[ s_{n+1,k} = s_{n,k-1} + n \cdot s_{n,k} \]

\item 2. Art \\
Anzahl der M\"oglichkeiten, genau $k$ nichtleere, paarweise disjunkte Partitionen aus \(n\) Elementen zu bilden.
\[ S_{n+1,k} = S_{n,k-1} + k \cdot S_{n,k} \]
\[ S_{n,k} = \frac{1}{k!} \sum^k_{j=0}(-1)^{k-j}{r \choose j}j^n \]
\end{itemize}

\subsubsection{Eulersche Zahlen}

Anzahl der \(n\)-Permutation, wo genau \(k\) Elemente gr"o"ser als ihr 
vorhergehendes sind.
\[ E_{n,k} = (n-k) \cdot E_{n-1,k-1} + (k+1) \cdot E_{n-1,k} \]

%\subsection{Longest Increasing Subsequence}

%\lstinputlisting{rekurrenzen/lis.cpp}

%\subsection{\textsc{SUBSET-SUM} (\textsc{KNAPSACK} nur mit Gewichten)}

%\lstinputlisting{rekurrenzen/subset-sum.cpp}

\subsection{Nim (bisschen Spieltheorie)}

Gegeben: \(n\) Haufen mit \(a_1,\ldots,a_n\) Steinen, wer dran ist nimmt bis zu \(k\) Steine von einem Haufen (\(k=\infty\) erlaubt).

\begin{description}
\item[1. Variante:] Ziel ist es als letztes einen Stein aufzunehmen. Man gewinnt, wenn
\lstinline{(a[1] % (k + 1)) XOR ... XOR (a[n] % (k + 1)) != 0} vor dem eigenen Zug.
\item[2. Variante:] Ziel ist es, dass der Gegner den letztes Stein aufnimmt:
To win at mis\`ere nim, play exactly as if you were playing normal play nim, except if your winning move would
lead to a position that consists of heaps of size one only. In that case, leave exactly one more or one fewer
heaps of size one than the normal play strategy recommends.
\item[3. Variante:] Ziel ist es, den letzten Stein eines der Haufen aufzunehmen. Man gewinnt, wenn
vor dem eigenem Zug ein Stapel maximal \(k\) Steine hat oder
\lstinline{(a[1] % (k + 1)) XOR ... XOR (a[n] % (k + 1)) != 0},
indem man entweder den letzten Stein nimmt oder man genau diesen Zustand herstellt.
\end{description}

%\subsection{Optimaler Suchbaum}
%Mit Knuths Verbesserung.

%max-sum-2d

%%% Local Variables:
%%% TeX-master: "tcr"
%%% End:
